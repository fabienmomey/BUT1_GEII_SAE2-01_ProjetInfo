\documentclass[10pt, fleqn, a4paper]{article}
%% \documentclass[11pt,draftcls,onecolumn]{IEEEtran}

\usepackage[left=2cm,right=2cm,top=1cm,bottom=2cm]{geometry}
\usepackage[utf8]{inputenc}
\usepackage{amsmath,amssymb,amsfonts}
\usepackage{graphicx}
\graphicspath{{../Figures/}}
\usepackage{hyperref}
\usepackage{xspace}
\usepackage{enumitem}
\usepackage{ulem}
\usepackage{cancel}
\usepackage{verbatim}
\usepackage{rotating}
\usepackage{array}
\usepackage{multirow}
\usepackage{color}
\usepackage[french]{babel}
\frenchbsetup{StandardItemLabels}
\usepackage{comment}
\usepackage{eurosym}
\usepackage{xfrac}
\usepackage{manfnt}
\usepackage{pifont}
\usepackage{tcolorbox}
\usepackage{textcomp}

\newcommand\prevyear{\advance\year by -1 \the\year\advance\year by 1}
\newcommand\nextyear{\advance\year by 1 \the\year\advance\year by -1}

% Empty definitions for further redefinitions via \renewcommand:
\newcommand{\Temp}{}
%% Abbreviations communes
\newcommand{\cf}{\emph{cf.}\xspace}
\newcommand{\eg}{\emph{e.g.}\xspace}
\newcommand{\ie}{\emph{i.e.}\xspace}
\newcommand{\etc}{\emph{etc.}\xspace}
\newcommand{\codeblocks}{Code$::$Blocks\xspace}

\newcommand{\touche}[1]{\textnormal{\texttt{\textquotesingle #1\textquotesingle}}}

\newenvironment{algorithme}
{
\begingroup

\footnotesize

\hrulefill

}
{

\hrulefill

\endgroup
}

\definecolor{green}{rgb}{0.0,0.5,0.0}
\definecolor{dred}{rgb}{0.75,0.0,0.0}
\definecolor{purple}{rgb}{0.75,0.0,0.75}
\definecolor{dblue}{rgb}{0.0,0.0,0.75}
\definecolor{dyellow}{rgb}{1.0,0.5,0.0}
\definecolor{orange}{rgb}{1.0,0.5,0.0}
\definecolor{redCM}{rgb}{1.0,0.25,0.25}
\definecolor{blueTD}{rgb}{0.25,0.75,1.0}
\definecolor{greenTP}{rgb}{0.1,0.75,0.5}
\definecolor{lightgreen}{rgb}{0.25,1.0,0.75}
\definecolor{mygray}{gray}{0.45}
\definecolor{redC1}{rgb}{1.0,0.25,0.25}
\definecolor{greenC2}{rgb}{0.0,0.75,0.0}
\definecolor{yellowPORTFOLIO}{rgb}{0.75,1.0,0.0}

\newtcbox{\cboxgenerale}{colback=orange,boxrule=1pt,arc=5pt,
  boxsep=0pt,left=3pt,right=3pt,top=3pt,bottom=3pt}

\newtcbox{\cboxconcevoir}{colback=redC1,boxrule=1pt,arc=5pt,
  boxsep=0pt,left=3pt,right=3pt,top=3pt,bottom=3pt}
  
\newtcbox{\cboxverifier}{colback=greenC2,boxrule=1pt,arc=5pt,
  boxsep=0pt,left=3pt,right=3pt,top=3pt,bottom=3pt}
  
\newtcbox{\cboxportfolio}{colback=yellowPORTFOLIO,boxrule=1pt,arc=5pt,
  boxsep=0pt,left=3pt,right=3pt,top=3pt,bottom=3pt}
  
\newtcbox{\cboxjalon}{colback=orange,boxrule=1pt,arc=5pt,
  boxsep=0pt,left=3pt,right=3pt,top=3pt,bottom=3pt}

\newcommand{\itemcolor}[1]{% Update list item colour
  \renewcommand{\makelabel}[1]{\color{#1}\hfil ##1}}
  
\newcommand{\bfcolor}[2]{\textcolor{#1}{\textbf{#2}}}

\newcommand{\itcolor}[2]{\textcolor{#1}{\textit{#2}}}

\newcommand{\trace}{\fbox{\colorbox{orange}{TRACE}}\xspace}

\newsavebox{\mytextbox}
\newcommand\myframecolor{}
\newcommand\mybgcolor{}
\newenvironment{mycolorbox}[2]
{
\def\myframecolor{#1}
\def\mybgcolor{#2}
\begingroup
\begin{lrbox}{\mytextbox}
\begin{minipage}[t]{\textwidth}
}
{
\end{minipage}\end{lrbox}
\fcolorbox{\myframecolor}{\mybgcolor}{\usebox{\mytextbox}}
\endgroup
}

\def\ESPACEINFO1{\href{https://claroline-connect.univ-st-etienne.fr/web/app.php/workspaces/41547/open?toolName=home}{\bfcolor{orange}{IUTSEGEII\_S1\_M1103\_INFO1}}}

\def\TABLEAUEVAL{\href{https://docs.google.com/spreadsheets/d/18LC7yuCl2cteuaYs7VfMBd0qE6IhRl_J6CCGcIUdCyk/edit?usp=sharing}{\bfcolor{orange}{Tableau de suivi et d'évaluation}}}

% Uncomment to compile doc with corrections
%\renewenvironment{comment}{}{}

\parindent=0cm

\newcounter{quest}

\begin{document}

\begin{minipage}[l]{\textwidth}
\begin{center}
\includegraphics[width=0.2\textwidth]{logoIUT} \\
\large IUT, Université Jean Monnet, Saint-Étienne \\
\vspace{0.5cm}
\huge \bf{SAé 2 Info2 : Projet d'Algorithmique et Programmation informatique} \\
\vspace{0.2cm}
\Large \bfcolor{blueTD}{DOCUMENT DE SUIVI ET D'ÉVALUATION} \\
\vspace{0.2cm}
\Large Année \prevyear-\the\year \\ %-\nextyear \\
\vspace{0.5cm}
\large Fabien Momey, Bruno Bernard, Paul Grandamme, Franck Gérossier, Vincent Grosso, Cyril Mauclair, Thierry Lagniet \\
\end{center}
\end{minipage}

\vspace{0.5cm}

\fbox{\begin{minipage}[l]{\textwidth}
{\bf \underline{OBJET DU PRÉSENT DOCUMENT~:}}

\vspace{1mm}
Ce document sert de référence pour le suivi et l'évaluation des groupes projet par leurs enseignants-évaluateurs respectifs.

Il contient notamment, pour chaque Jalon-Livrable, les critères d'évaluation et barèmes associés, afin d'être homogènes dans les points à vérifier et à évaluer (mis en regard des compétences).
\end{minipage}}

\vspace{0.5cm}

\fbox{\begin{minipage}[l]{\textwidth}
{\bf \underline{RESSOURCES À DISPOSITION~:}}

Les ressources dédiées aux enseignants-évaluateurs pour assurer le suivi se trouvent sur \ESPACEINFO1. Un onglet \bfcolor{green}{SAé2 Info2 - Ressources Enseignants-Évaluateurs} a été créé dans lequel vous retrouverez le \bfcolor{greenTP}{présent document} ainsi que des \bfcolor{greenTP}{ressources dédiées} \itcolor{greenTP}{(à consulter régulièrement)}.

\vspace{2mm}
{\bf \underline{\textit{Résumé des notions à maîtriser par les étudiants}~:}}
\begin{itemize}
\item[$\looparrowright$] \bfcolor{orange}{Connaissances techniques :}
\begin{itemize}
\item[\ding{223}] \colorbox{yellow}{\bfcolor{orange}{Bases d'algorithmique et programmation informatique.}}
\item[\ding{223}] Variables, types, structures de contrôle (conditions, boucles).
\item[\ding{223}] Tableaux, \colorbox{yellow}{\bfcolor{orange}{structures}}, \colorbox{yellow}{\bfcolor{orange}{pointeurs}}.
\item[\ding{223}] \colorbox{yellow}{\bfcolor{orange}{Fonctions}}~: appel de fonctions, définition de fonctions, notion de \bfcolor{orange}{passage par valeur (entrées en \og{}lecture seule\fg{} E)} et \bfcolor{orange}{passage par adresse (entrées/sorties en \og{}lecture/écriture\fg{} E/S)}.
\item[\ding{223}] \bfcolor{orange}{\textit{Allocation dynamique de mémoire.}}
\end{itemize}
\item[$\looparrowright$] \bfcolor{redC1}{Mise en œuvre d'une solution algorithmique complète exploitant ces connaissances.}
\item[$\looparrowright$] \bfcolor{redC1}{Implémentation d'une solution algorithmique complète dans un langage de programmation (C++).}
\item[$\looparrowright$] Structure d’un programme informatique :
\begin{itemize}
\item[\ding{223}] fichier source principal "\texttt{main.cpp}".
\item[\ding{223}] fichiers d'en-tête "\texttt{.h}" : prototypes des fonctions et déclaration des structures.
\item[\ding{223}] fichiers sources "\texttt{.cpp}" : définition des fonctions.
\end{itemize}
\item[$\looparrowright$] Écriture du code et compilation dans un environnement de développement intégré (IDE) :
\begin{itemize}
\item[\ding{223}] Code$::$Blocks.
\end{itemize}
\item[$\looparrowright$] \bfcolor{greenC2}{Vérification du code} à l'aide de \bfcolor{greenC2}{procédures de test} et \bfcolor{greenC2}{débogage}.
\end{itemize}
\end{minipage}}

\vspace{0.5cm}

\fbox{\begin{minipage}[l]{\textwidth}
{\bf \underline{COMPÉTENCES \& APPRENTISSAGES CRITIQUES MIS EN JEU~:}}

\begin{center}
\cboxconcevoir{C1-N1-CONCEVOIR}
\end{center}

\cboxconcevoir{\textit{C1-N1-AC1 - Produire une analyse fonctionnelle d'un système simple}}

\begin{itemize}
\item[\ding{223}] Pour toutes les étapes de ce projet, les étudiants doivent proposer une solution algorithmique complète et détaillée (découpage fonctionnel, algorithme principal, procédures de test), indépendamment du langage de programmation choisi, qui respecte les contraintes imposées dans le cahier des charges.
\end{itemize}

\cboxconcevoir{\textit{C1-N1-AC2 - Réaliser un prototype pour des solutions techniques matériel ou logiciel}}

\begin{itemize}
\item[\ding{223}] Les étudiants doivent ensuite implémenter dans une application console C++ chaque étape du projet, en traduisant fidèlement la solution algorithmique élaborée.
\end{itemize}

\cboxconcevoir{\textit{C1-N1-AC3 - Rédiger un dossier de fabrication à partir d'un dossier de conception}}

\begin{itemize}
\item[\ding{223}] Les étudiants doivent rédiger une \bfcolor{redCM}{documentation complète} de leur programme, sous forme algorithmique et illustrée par des exemples issues de leur prototypage en C++. \bfcolor{redCM}{Ce rapport doit pouvoir servir de référence à un programmeur tierce qui souhaiterait implémenter leur solution dans un autre langage de programmation}.
\item[\ding{223}] Concernant leur programme C++, ce dernier devra aussi être \bfcolor{redCM}{rigoureusement commenté}, et notamment toutes les fonctions avec leurs \textbf{\{ Rôle, Entrées, Entrées/Sorties, Sortie \}} clairement définis.
\end{itemize}

\begin{center}
\cboxverifier{C2-N1\textcolor{yellow}{(N3)}-VERIFIER}
\end{center}

\cboxverifier{\textit{C2-N1-AC1 - Appliquer une procédure d’essai}}

\begin{itemize}
\item[\ding{223}] Pour chaque étape du projet, lors de la phase de conception algorithmique et d'implémentation, les étudiants doivent réfléchir à une batterie de tests unitaires à mettre en place - \textbf{sous forme de procédures à implémenter ou à réaliser avec l'outil \bfcolor{greenC2}{débogueur} par exemple} - pour valider chaque élément de leur solution (programme principal, fonctions, structures, etc.). \bfcolor{redCM}{Toutes les procédures de test devront être rigoureusement décrites et documentées dans leur rapport. Si ce sont des procédures de test implémentées, elles devront être intégrées au code (dans une librairie dédiée par exemple) et rigoureusement commentées}.
\end{itemize}

\cboxverifier{\textit{C2-N1-AC2 - Identifier un dysfonctionnement}}

\begin{itemize}
\item[\ding{223}] L'application des procédures d'essai que les étudiants auront définies doivent leur permettre de vérifier le bon fonctionnement de leur programme et d'identifier des dysfonctionnements éventuels.
\end{itemize}

\cboxverifier{\textit{C2-N1-AC3 - Décrire les effets d’un dysfonctionnement}}

\begin{itemize}
\item[\ding{223}] En cas d'identification d'un dysfonctionnement, les étudiants doivent garder trace de leurs tests et analyses en rédigeant une petite description dans leur \colorbox{yellowPORTFOLIO}{journal de bord}.
\end{itemize}

\cboxverifier{\color{yellow}\textit{C2-N3-AC2 - Proposer une solution corrective à un dysfonctionnement}}

\begin{itemize}
\item[\ding{223}] Bien entendu, tout dysfonctionnement de leur programme devra être corrigé et l'application de leur procédure d'essai devra permettre de valider la modification. Cette dernière sera également notifiée dans leur \colorbox{yellowPORTFOLIO}{journal de bord}.
\end{itemize}
\end{minipage}}

\vspace{0.5cm}

\fbox{\begin{minipage}[l]{\textwidth}
{\bf \underline{JALONS \& LIVRABLES~:}}

Tout au long de l'élaboration des phases du projet, un certain nombre de jalons seront à respecter, avec des dates fixées et des livrables à fournir (codes, journaux de bord, état de la documentation). \bfcolor{redCM}{Ces livrables seront à déposer sur \ESPACEINFO1 dans l'onglet dédié \colorbox{yellow}{\color{red}SAé2 Info2 - Livrables pour le projet d'informatique}}. 

Les jalons sont indiqués par le cartouche suivant à la fin des phases concernées~:

\begin{center}
\cboxjalon{\bf Jalon \& Livrable n$^{\circ}$\textit{i}}
\end{center}

Chacun de ces jalons sera évalué par les enseignants-évaluateurs, et des entrevues pourront être prévues.
\bfcolor{redCM}{Voici les dates butoires à respecter (les espace de dépôt sur \ESPACEINFO1 seront ouverts ces jours)~:}

\cboxjalon{\bf Jalon \& Livrable n$^{\circ}$1 $\Rightarrow$ 01/04/2022}

\cboxjalon{\bf Jalon \& Livrable n$^{\circ}$2 $\Rightarrow$ 15/04/2022}

\cboxjalon{\bf Jalon \& Livrable n$^{\circ}$3 $\Rightarrow$ 06/05/2022}

\cboxjalon{\bf Jalon \& Livrable n$^{\circ}$4 $\Rightarrow$ 24/05/2022}

\cboxjalon{\bf Jalon \& Livrable n$^{\circ}$5 $\Rightarrow$ 10/06/2022}

\end{minipage}}

\vspace{0.5cm}

\fbox{\begin{minipage}[l]{\textwidth}
{\bf \underline{CONVENTIONS DE CODAGE À RESPECTER~:}}
\begin{itemize}
\item[$\looparrowright$] Uniquement des caractères alphanumériques pour le nommage des éléments de code (variables, fonctions, etc.). Pas de caractère accentué, pas d'espace, pas de point, seul le caractère \og{}underscore (tiret du 8)\fg{} est toléré.
\item[$\looparrowright$] les \textbf{variables} doivent être nommées en minuscule.\\
 \underline{\textit{Exemple~:}}\\
 \fbox{\texttt{int mavariable~;}}
 \item[$\looparrowright$] les \textbf{constantes} doivent être nommées en majuscule.\\
 \underline{\textit{Exemple~:}}\\
 \fbox{\texttt{const int MACONSTANTE~;}}
\item[$\looparrowright$] les \textbf{structures} doivent être nommées par un mot commençant par une majuscule.\\
 \underline{\textit{Exemple~:}}\\
 \fbox{\texttt{struct Mastruct \{ \}~;}}
\item[$\looparrowright$] les \textbf{champs} de structures doivent être nommées en minuscules et commencer par le préfixe \texttt{m\_}.
\\
 \underline{\textit{Exemple~:}}\\
 \fbox{\texttt{int m\_champ1~;}}
 \item[$\looparrowright$] les \textbf{fonctions} doivent nommées par un ou plusieurs mots attachés chacun commençant par une majuscule.\\
 \underline{\textit{Exemple~:}}\\
 \fbox{\texttt{void MaFonctionSuperGeniale()~;}}
 \item[$\looparrowright$] les \textbf{entrées} et \textbf{entrées/sorties} des fonctions doivent être nommées en minuscule et commencer par le préfix \texttt{input\_} pour les entrées et \texttt{output\_} pour les entrées/sorties.
\\
 \underline{\textit{Exemple~:}}\\
 \fbox{\texttt{void MaFonctionSuperGeniale(const int* input\_var1, float* output\_var2)~;}}\\
 \itcolor{redCM}{\underline{Note~:} les entrées (lecture seule) doivent être précédées du mot-clé \textnormal{\texttt{const}} pour bien les identifier comme étant en lecture seule ...}
\end{itemize}
\end{minipage}}

\newpage
\begin{mycolorbox}{black}{yellow}
\bf DANS LA SUITE DE CE DOCUMENT, POUR CHAQUE PHASE, ON DONNE LES POINTS À ÉVALUER ET LES CRITÈRES ASSOCIÉS LORS DE L'ÉVALUATION DU JALON CONCERNÉ.\\
CES CRITÈRES SONT RETRANSCRIS DANS LE \TABLEAUEVAL. CHAQUE CRITÈRE CORRESPOND À UNE COLONNE À NOTER AVEC UN POURCENTAGE TRADUISANT LE "NIVEAU DE RÉUSSITE" DE L'OBJECTIF VISÉ.
\end{mycolorbox}

\addtocounter{quest}{1}
\subsection*{Phase~\thequest : création du conteneur \texttt{Vaisseau} générique}
\label{phase_vaisseau}


\begin{itemize}
\item[$\looparrowright$] Points à vérifier pour respecter le \textbf{cahier des charges}~:
\begin{itemize}
\item[\ding{223}] Il faut créer une \textbf{structure} \texttt{Vaisseau} avec 4 champs : 2 chaînes de caractères et 2 entiers. Toute autre solution peut être acceptable en partie si elle est correctement justifiée et documentée mais à terme, ne pas utiliser une structure risque de rendre le code compliqué à gérer.
\item[\ding{223}] La librairie de fonctions doit permettre de manipuler facilement des \textbf{variables de type \texttt{Vaisseau}}. Le vaisseau "appelant" doit être passé en premier paramètre de la fonction en tant que pointeur. \itcolor{blue}{Le grosse difficulté va être la gestion des pointeurs sur structure dans les fonctions $\Rightarrow$ déréférencement puis accès au champ de la structure~:}
\verb!output\_vaisseau->m_nom! ou \verb!(*output\_vaisseau).m_nom!
\item[\ding{223}] En tout il devrait y avoir 1 fonction \texttt{\bf Afficher}, \texttt{\bf AffecterCarac} (affectation de \underline{tous} champs dans la même fonction), 4 fonctions \texttt{\bf Affecter\textit{Champ}} (1 par champ), 4 fonctions \texttt{\bf Renvoyer\textit{Champ}} (1 par champ). \itcolor{red}{Attention au respect de la "généricité" des fonctions par les étudiants qui ont tendance à "rajouter des fonctionnalités" dans ces fonctions qui doivent avoir un rôle unique. Ces fonctions doivent constituer les outils de base pour créer ensuite des fonctions plus complexes utilisant ces dernières.}
\item[\ding{223}] L'affectation des chaînes de caractères doit se faire préférentiellement à l'aide de la librairie \texttt{string.h} (fonction \texttt{\bf strcpy}). Attention également aux fonctions renvoyant des chaînes de caractères.
\item[\ding{223}] Les fonctions doivent nommées suivant une "nomenclature" commune et claire. L'idéal est de créer une librairie dédiée (\texttt{.h} et \texttt{.cpp}) avec un nommage clair : par exemple \texttt{lib\_vaisseau}.
\item[\ding{223}] \itcolor{blue}{Il faut bien comprendre que dans cette phase, on implémente surtout \underline{l'architecture} du jeu et non le jeu lui-même, c'est-à-dire toutes les fonctionnalités qui gérer des \textbf{"objets"} \texttt{Vaisseau}, à la manière du \textbf{paradigme orienté objet}.}
\item[\ding{223}] \bfcolor{red}{Interdiction d'utiliser des classes !!! Les étudiants ne connaissent pas ce paradigme et l'aborderont en deuxième année.}
\end{itemize}
\end{itemize}

\addtocounter{quest}{1}
\subsection*{Phase~\thequest : \og{}duel de vaisseaux\fg{}}
\label{phase_duel}

\begin{itemize}
\item[$\looparrowright$] Points à vérifier pour respecter le \textbf{cahier des charges}~:
\begin{itemize}
\item[\ding{223}] Il y a une fonction à rajouter à la librairie \texttt{lib\_vaisseau} : \texttt{\bf SubirDegatsCoque} qui permet d'ôter un nombre entier de points de vie au vaisseau "appelant". \itcolor{blue}{Il faut penser à gérer si la résistance de coque tombe en-dessous de 0.}
\item[\ding{223}] Le programme de duel entre 2 vaisseaux doit utiliser la librairie \texttt{moteur\_de\_jeu}.
\item[\ding{223}] \itcolor{blue}{Attention dans cette phase à ce que les étudiants respectent bien la simplicité de l'algorithme demandé et "ne partent pas dans tous les sens".}
\end{itemize}
\end{itemize}

\vspace{5mm}
\cboxjalon{\bf Jalon \& Livrable n$^\circ$1 $\Rightarrow$ 01/04/2022}
\begin{mycolorbox}{black}{orange}
Pour ce jalon, Les étudiants doivent fournir dans une archive \texttt{ZIP} intitulée\\ \colorbox{yellow}{\texttt{Jalon\_1\_\textit{n$^\circ$binome}.zip}}~:
\begin{itemize}
\item[$\looparrowright$] leur code fonctionnel et correctement documenté.
\item[$\looparrowright$] leurs journaux de bord, personnels et de groupe, au format numérique PDF.
\item[$\looparrowright$] \textbf{On ne regarde pas encore la partie "algorithmes" qui sera évaluée à partir du Livrable n$^\circ$2.}
\end{itemize}
\begin{mycolorbox}{black}{yellow}Si ils ne sont pas parvenus jusqu'à ce point, ils doivent rendre \textbf{leur projet en l'état} à la date prévue.\end{mycolorbox}
\end{mycolorbox}

\begin{mycolorbox}{black}{yellow}
\underline{CRITÈRES D'ÉVALUATION~:}\\
Ces critères sont retranscris dans le \TABLEAUEVAL. Chaque critère correspondant à une colonne dans lequel 
\begin{enumerate}
\item \colorbox{redC1}{C1-N1} (\textbf{éval. de groupe})~: respect du cahier des charges.
\item \colorbox{redC1}{C1-N1} (\textbf{éval. de groupe})~: qualité d'implémentation de la structure + librairie \texttt{lib\_vaisseau} (phase 1)~: code clair, commenté.
\item \colorbox{redC1}{C1-N1} (\textbf{éval. de groupe})~: qualité d'implémentation du programme de duel de vaisseaux~(phase 2): code clair, commenté.
\item \colorbox{greenC2}{C2-N1} (\textbf{éval. de groupe})~: implémentation de procédures ou programmes de tests unitaires clairs et commentés.
\item \colorbox{yellowPORTFOLIO}{Tenue du journal de bord \textbf{de groupe} (\textbf{éval. de groupe})~:}
\begin{enumerate}
\item \colorbox{redC1}{C1-N1} $\Rightarrow$ les choix faits, la répartition des tâches, les tâches réalisées, les notes techniques.
\item \colorbox{greenC2}{C2-N1} $\Rightarrow$ la description des tests réalisés, l'identification de dysfonctionnements et solutions correctives.
\end{enumerate}
\item \colorbox{yellowPORTFOLIO}{Tenue du journal de bord \textbf{personnel} (\textbf{éval. individuelle})~:}
\begin{enumerate}
\item \colorbox{redC1}{C1-N1} $\Rightarrow$ les tâches réalisées, les notes techniques.
\item \colorbox{greenC2}{C2-N1} $\Rightarrow$ la description des tests réalisés, l'identification de dysfonctionnements et solutions correctives.
\end{enumerate}
\item \colorbox{redC1}{C1-N1}/\colorbox{greenC2}{C2-N1} implication du groupe dans le projet (\textbf{éval. de groupe})~: questions aux tuteurs, demandes de rendez-vous.
\item \colorbox{redC1}{C1-N1}/\colorbox{greenC2}{C2-N1} implication personnelle de l'étudiant dans le projet (\textbf{éval. individuelle})~: travail, participation dans les rendez-vous tuteurs.
\end{enumerate}
\end{mycolorbox}

%
%\addtocounter{quest}{1}
%\subsection*{Phase~\thequest : fonctionnalité \texttt{Attaquer}}
%\label{phase_attaquer}
%
%\begin{itemize}
%\item[$\looparrowright$] On souhaite à présent définir une fonctionnalité \texttt{Attaquer} \og{}appelée par un vaisseau attaquant (l'appelant)\fg{} qui va infliger des dégâts (la cible va donc \textbf{subir des dégâts}) du nombre de sa \textbf{puissance de feu}.\\
%\itcolor{redCM}{\underline{Note~:} cette fonctionnalité doit pouvoir utiliser des fonctionnalités déjà implémentées afin de favoriser l'intégrité de votre code.} 
%\end{itemize}
%
%\addtocounter{quest}{1}
%\subsection*{Phase~\thequest : modification du conteneur \texttt{Vaisseau} et adaptation du duel}
%\label{phase_type_vaisseau}
%
%\begin{itemize}
%\item[$\looparrowright$] On doit modifier le nom de la fonctionnalité \texttt{Attaquer} en \texttt{AttaquerBasic} qui constituera la \bfcolor{blueTD}{fonctionnalité d'attaque standard} du \texttt{Vaisseau}.
%\item[$\looparrowright$] En prévision des phases suivantes, on crée à présent une nouvelle fonctionnalité \texttt{AttaquerSpecial} qui constituera une \bfcolor{blueTD}{méthode d'attaque spécialisée} du \texttt{Vaisseau}. Pour le moment, \bfcolor{blueTD}{conteneur \texttt{Vaisseau} que vous avez implémenté constitue une classe générique de vaisseau}, qui ne possède en soi aucune spécialisation. La fonctionnalité \texttt{AttaquerSpecial} \bfcolor{blueTD}{ne doit donc rien faire hormis afficher un message spécifique}.\\
%\itcolor{redCM}{C'est lors des phases suivantes du projet que seront définies des \textbf{classes spécialisées de vaisseaux}, pour lesquelles la fonctionnalité \textnormal{\texttt{AttaquerSpecial}} devra être redéfinie pour permettre de réaliser des actions d'attaque spécifiques.} 
%\item[$\looparrowright$] On doit à présent \bfcolor{greenTP}{adapter le programme de duel de vaisseaux} précédemment programmé dans la \bfcolor{greenTP}{phase 2} afin qu'il utilise les nouvelles fonctionnalités \textnormal{\texttt{AttaquerBasic}} et \textnormal{\texttt{AttaquerSpecial}}.\\
%\itcolor{redCM}{Il faut alors intégrer un système (de votre choix, aléatoire par exemple) de sélection de l'attaque basique ou de l'attaque spéciale lors d'une itération du duel.}
%\end{itemize}
%
%\begin{mycolorbox}{black}{yellow}
%\itcolor{redCM}{\underline{Note~:}} cette adaptation doit être implémentée dans un nouveau programme principal, afin de garder trace de l'ancien programme. Vous pouvez créer un nouveau projet \codeblocks et y importer vos librairies de fonctions pour y implémenter un nouveau \texttt{main}.
%\end{mycolorbox}
%
%\vspace{5mm}
%\cboxjalon{\bf Jalon \& Livrable n$^\circ$2 $\Rightarrow$ 15/04/2022}
%\begin{mycolorbox}{black}{orange}
%Pour ce jalon, vous devrez fournir dans une archive \texttt{ZIP} intitulée\\ \colorbox{yellow}{\texttt{Jalon\_2\_\textit{n$^\circ$binome}.zip}}~:
%\begin{itemize}
%\item[$\looparrowright$] Votre code fonctionnel et correctement documenté.
%\item[$\looparrowright$] Vos journaux de bord, personnel et de groupe, au format numérique PDF.
%\end{itemize}
%\begin{mycolorbox}{black}{yellow}Si vous n'êtes pas parvenus jusqu'à ce point, vous devez rendre \textbf{votre projet en l'état} à la date prévue.\end{mycolorbox}
%\end{mycolorbox}
%
%\addtocounter{quest}{1}
%\subsection*{Phase~\thequest : définition de 2 conteneurs \og{}spécialisés\fg{} du \texttt{Vaisseau} : \texttt{Croiseur} et \texttt{Chasseur}}
%\label{phase_heritage}
%
%On propose de créer des \bfcolor{blueTD}{vaisseaux de classes spécialisées}, c'est-à-dire des vaisseaux qui vont intégrer, en plus des caractéristiques génériques à tout vaisseau, des caractéristiques et fonctionnalités particulières afin d'introduire un peu de variabilité et de personnalisation dans la future constitution de \bfcolor{blueTD}{flottes de vaisseaux}.\\
%\bfcolor{redCM}{\underline{Comment faire ?}} Il s'agit de créer des nouveaux conteneurs \og{}spécialisés\fg{} qui doivent \og{}hériter\fg{} des caractéristiques d'un conteneur \texttt{Vaisseau}. Pour éviter de \og{}recopier\fg{} les caractéristiques génériques dans ces nouveaux conteneurs, il faut globalement inclure un conteneur de type \texttt{Vaisseau} comme une caractéristique des conteneurs spécialisés $\Rightarrow$ on aura ainsi \og{}encapsuler\fg{} toutes les caractéristiques génériques et on pourra avoir accès aux fonctionnalités implémentées pour gérer cette caractéristique \texttt{Vaisseau} intégrée au conteneur spécialisé. 
%
%L'objectif de cette phase est donc de~:
%\begin{itemize}
%\item[$\looparrowright$] créer 2 conteneurs - \itcolor{redCM}{2 nouveaux \textbf{types} de vaisseaux} - \texttt{Croiseur} et \texttt{Chasseur} qui doivent \og{}spécialiser\fg{} le conteneur \texttt{Vaisseau}. Pour le moment ces conteneurs doivent être \bfcolor{blueTD}{vides} hormis la caractéristique \texttt{m\_vaisseau} définissant ce lien \og{}d'héritage\fg{} d'un type \texttt{Vaisseau} dans ces conteneurs.\\
%\itcolor{redCM}{D'un point de vue logique, un \textnormal{\texttt{Chasseur}} ou un \textnormal{\texttt{Croiseau}} \textbf{est} un \textnormal{\texttt{Vaisseau}}. D'un point de vue implémentationnel, on le modélise par le fait qu'un \textnormal{\texttt{Chasseur}} ou un \textnormal{\texttt{Croiseau}} \textbf{contient} un \textnormal{\texttt{Vaisseau}}.}
%\item[$\looparrowright$] adapter le programme de duel de vaisseaux précédemment programmé dans la \bfcolor{greenTP}{phase 4} afin de tester un duel entre 1 \texttt{Chasseur} et 1 \texttt{Croiseur}.\\
%\itcolor{redCM}{L'interaction entre 2 vaisseaux spécialisés doit utiliser les fonctionnalités du type générique \texttt{Vaisseau} - \textnormal{\texttt{AttaquerBasic}} et \textnormal{\texttt{AttaquerSpecial}}. Il faut donc trouver un moyen pour que des vaisseaux spécialisés puissent \og{}appeler\fg{} ces fonctionnalités.}
%\end{itemize}
%
%\begin{mycolorbox}{black}{yellow}
%\itcolor{redCM}{\underline{Note~:}} une fois de plus, cette adaptation doit être implémentée dans un nouveau programme principal, afin de garder trace des anciens programmes.
%\end{mycolorbox}
%
%\addtocounter{quest}{1}
%\subsection*{Phase~\thequest : spécialisation du \texttt{Croiseur}}
%\label{phase_croiseur}
%
%La spécificité du \texttt{Croiseur} est de disposer d'une \bfcolor{blueTD}{arme spécifique}, le \bfcolor{blueTD}{canon}, matérialisée par 1 nouvelle caractéristique~:
%\begin{itemize}
%\item[\ding{223}] la puissance de feu du canon.
%\end{itemize}
%
%La spécialisation du \texttt{Croiseur} consiste en~:
%\begin{itemize}
%\item[$\looparrowright$] l'ajout des nouvelles caractéristiques au conteneur \texttt{Croiseur}.
%\item[$\looparrowright$] la définition de nouvelles fonctionnalités permettant d'\textbf{affecter} (globalement et individuellement) des valeurs à aux caractéristiques spécifiques d'un croiseur.
%\item[$\looparrowright$] la re-définition de la fonctionnalité permettant d'\textbf{afficher} les caractéristiques d'un croiseur.\\
%\itcolor{blueTD}{On devra donc afficher aussi bien les nouvelles caractéristiques que les caractéristiques génériques (pour ces derniers on devra ré-utiliser la fonctionnalité \texttt{Afficher} du conteneur \texttt{Vaisseau}).}\\
%\itcolor{redCM}{\underline{Note :} d'un point de vie implémentationnel, la re-définition de fonctionnalité va faire appel à la technique de \textbf{surcharge de fonction}. Cette dernière consiste à redéfinir une fonction avec le même nom qu'une autre mais des paramètres d'entrée différents. Ainsi dans notre cas, il s'agit d'adapter la fonctionnalité \textnormal{\texttt{Afficher}} du conteneur \textnormal{\texttt{Vaisseau}} pour que l'\og{}appelant\fg{} soit un \textnormal{\texttt{Croiseur}}.}
%\end{itemize}
%\begin{mycolorbox}{black}{yellow}
%\bfcolor{greenC2}{Il ne faut pas oublier de réaliser des procédures de tests unitaires pour valider chaque nouvel élément ajouté au projet.}
%\end{mycolorbox}
%\begin{itemize}
%\item[$\looparrowright$] la re-définition (sur le même principe que la fonctionnalité \texttt{Afficher}) de la fonctionnalité \texttt{AttaquerSpecial} qui va infliger les dégâts d'un \bfcolor{blueTD}{tir de canon}.\\
%\bfcolor{greenTP}{Cette attaque spéciale ne doit avoir qu'1 chance sur 2 de réussir.}
%\end{itemize}
%
%\begin{mycolorbox}{black}{yellow}
%\itcolor{redCM}{\underline{Note~:}} Si tout est implémenté correctement, le programme de duel entre un croiseur et un chasseur de la phase 5 doit fonctionner correctement sans apporter de modification.
%\end{mycolorbox}
%
%\addtocounter{quest}{1}
%\subsection*{Phase~\thequest : spécialisation du \texttt{Chasseur}}
%\label{phase_escadron}
%
%La spécificité du \texttt{Chasseur} est de disposer d'une \bfcolor{blueTD}{arme spécifique}, les \bfcolor{blueTD}{torpilles}, matérialisée par 2 nouvelles caractéristiques~:
%\begin{itemize}
%\item[\ding{223}] la puissance de feu des torpilles~;
%\item[\ding{223}] le \og{}stock\fg{} disponible de torpilles (qui sera donc décrémenté à chaque utilisation et finira par tomber à zéro).
%\end{itemize}
%
%La spécialisation du \texttt{Chasseur} suit les mêmes étapes que la spécialisation du \texttt{Croiseur}. La seule différence sera dans la re-définition de la fonctionnalité \texttt{AttaquerSpecial} \bfcolor{greenTP}{qui aura 100\% de chances de réussite mais sera limitée en nombre d'utilisation, géré par l'une des caractéristiques spéciales du \texttt{Chasseur}.}
%
%\vspace{5mm}
%\cboxjalon{\bf Jalon \& Livrable n$^\circ$3 $\Rightarrow$ 06/05/2022}
%\begin{mycolorbox}{black}{orange}
%Pour ce jalon, vous devrez fournir dans une archive \texttt{ZIP} intitulée\\ \colorbox{yellow}{\texttt{Jalon\_3\_\textit{n$^\circ$binome}.zip}}~:
%\begin{itemize}
%\item[$\looparrowright$] Votre code fonctionnel et correctement documenté.
%\item[$\looparrowright$] Une première version de la documentation technique du code, au format numérique PDF.
%\item[$\looparrowright$] Vos journaux de bord, personnel et de groupe, au format numérique PDF.
%\end{itemize}
%\begin{mycolorbox}{black}{yellow}Si vous n'êtes pas parvenus jusqu'à ce point, vous devez rendre \textbf{votre projet en l'état} à la date prévue.\end{mycolorbox}
%\end{mycolorbox}
%
%\addtocounter{quest}{1}
%\subsection*{Phase~\thequest : consolidation du moteur du programme}
%\label{phase_surcharge}
%
%Pour le moment, les programmes de duel entre 2 vaisseaux sont conçus spécifiquement pour gérer un combat entre 2 vaisseaux dont on connait la classe~:
%\begin{itemize}
%\item[\ding{223}] Un \texttt{Vaisseau} contre un \texttt{Vaisseau}~;
%\item[\ding{223}] Un \texttt{Croiseur} contre un \texttt{Chasseur}~;
%\end{itemize}
%
%On souhaite pouvoir généraliser le programme de duel pour que son \textbf{algorithme principal} soit le même quels que soient les 2 vaisseaux \og{}duellistes\fg{}, autrement dit quels que soient les 2 vaisseaux déclarés dans le \textbf{lexique principal}.
%
%\begin{itemize}
%\item[$\looparrowright$] Pour ce faire, il faut rendre \textbf{génériques} les appels aux fonctionnalités \texttt{AttaquerBasic} et \texttt{AttaquerSpecial} pour que la compilation du programme \og{}détecte automatiquement\fg{} les classes de vaisseaux antagonistes~:
%\begin{itemize}
%\item[\ding{223}] Un \texttt{Vaisseau} attaque un \texttt{Croiseur}~;
%\item[\ding{223}] Un \texttt{Croiseur} attaque un \texttt{Vaisseau}~;
%\item[\ding{223}] Un \texttt{Chasseur} attaque un \texttt{Chasseur}~;
%\item[\ding{223}] Un \texttt{Croiseur} attaque un \texttt{Croiseur}~;
%\item[\ding{223}] etc.
%\end{itemize}
%On va procéder en \bfcolor{redCM}{surchargeant} ces fonctionnalités en autant de versions que de cas possibles.
%\item[$\looparrowright$] On peut alors générer autant d'exécutables du programme de duel en ne changeant avant compilation que la déclaration des 2 vaisseaux antagonistes.
%\end{itemize}
%
%\cboxjalon{\bf Jalon \& Livrable n$^\circ$4 $\Rightarrow$ 24/05/2022}
%
%\cboxjalon{\bf OBJECTIF MINIMAL DU PROJET}
%
%\begin{mycolorbox}{black}{orange}
%Il \underline{faut} arriver à rendre ce code fonctionnel, et tout doit être finalisé pour le rendu final du projet.
%
%Pour ce jalon, vous devrez fournir dans une archive \texttt{ZIP} intitulée\\ \colorbox{yellow}{\texttt{Jalon\_4\_\textit{n$^\circ$binome}.zip}}~:
%\begin{itemize}
%\item[$\looparrowright$] Votre code fonctionnel et correctement documenté.
%\item[$\looparrowright$] La documentation technique du code, au format numérique PDF.
%\item[$\looparrowright$] Vos journaux de bord, personnel et de groupe, au format numérique PDF.
%\end{itemize}
%\begin{mycolorbox}{black}{yellow}Si vous n'êtes pas parvenus jusqu'à ce point, vous devez rendre \textbf{votre projet en l'état} à la date prévue.\end{mycolorbox}
%\end{mycolorbox}
%
%\addtocounter{quest}{1}
%\subsection*{Phase~\thequest : création d'un conteneur \texttt{Flotte}}
%\label{phase_flotte}
%
%\begin{itemize}
%\item[$\looparrowright$] L'objectif final est de pouvoir gérer une \textbf{bataille globale entre 2 flottes de vaisseaux}. Pour ce faire, on propose de créer un conteneur \texttt{Flotte} qui contiendra plusieurs vaisseaux~: 
%\begin{itemize}
%\item[\ding{223}] 1 \texttt{Croiseur}~;
%\item[\ding{223}] 2 \texttt{Chasseur}~;
%\item[\ding{223}] 3 \texttt{Vaisseau}.
%\itcolor{redCM}{\underline{Note :} ces vaisseaux doivent être regroupés par type, autrement dit une caractéristique du conteneur \texttt{Flotte} correspond à un groupement de vaisseaux d'un certain type $\Rightarrow$ \textbf{le type approprié est donc à déterminer}.}
%\end{itemize}
%\item[$\looparrowright$] Une instance de \texttt{Flotte} doit disposer des fonctionnalités suivantes~:
%\begin{itemize}
%\item[\ding{223}] 1 fonctionnalité permettant d'\textbf{afficher} l'état de la \texttt{Flotte} : informations sur chaque \texttt{Vaisseau}. Cette fonctionnalité \bfcolor{blueTD}{doit entre autres appeler les fonctionnalités \texttt{Afficher} de chaque \texttt{Vaisseau}, \texttt{Croiseur} et \texttt{Chasseur} de la flotte}.
%\item[\ding{223}] plusieurs fonctionnalités (1 par type de vaisseau) permettant de \bfcolor{blueTD}{modifier les caractéristiques d'un \texttt{Vaisseau}, \texttt{Croiseur} ou \texttt{Chasseur} de la flotte}.\\
%\itcolor{redCM}{\underline{Note :} la désignation d'un \textnormal{\tt Vaisseau}, \textnormal{\tt Croiseur} ou \textnormal{\tt Chasseur} au sein de la \textnormal{\tt Flotte} doit être géré par un \textbf{numéro d'ordre}~:
%\begin{itemize}
%\item $\{1, 2, 3\}$ pour l'un des 3 \textnormal{\tt Vaisseau}.
%\item $\{4, 5\}$ pour l'un des 2 \textnormal{\tt Chasseur}.
%\item $\{6\}$ pour le \textnormal{\tt Croiseur}.
%\end{itemize}
%Ce \textbf{numéro d'ordre} (type \textnormal{\tt int}) sera un des paramètres d'entrée des fonctionnalités \textbf{d'affectation}, afin de désigner le bon vaisseau.}
%\item[\ding{223}] 1 fonctionnalité qui renvoie \texttt{VRAI} si l'ensemble des vaisseaux de la flotte sont détruits, et \texttt{FAUX} sinon.
%\end{itemize}
%\item[$\looparrowright$] Ce nouveau conteneur étant assez conséquent, un soin particulier doit être porté aux \bfcolor{greenC2}{procédures de tests unitaires} afin de valider rigoureusement le bon fonctionnement de ce conteneur.\\
%\end{itemize}
%
%\addtocounter{quest}{1}
%\subsection*{Phase~\thequest : l'ultime bataille}
%\label{phase_bataille1}
%
%\begin{itemize}
%\item[$\looparrowright$] Nous voici enfin à l'objectif final de ce projet~: \textbf{la mise en place d'un \og{}jeu de combat\fg{} entre 2 \texttt{Flotte} au tour par tour}. Voici l'algorithme simplifié à coder dans votre programme principal~:
%
%\begin{algorithme}
%Création et initialisation des flottes.\\
%\textbf{Répéter}
%\begin{equation*}
%\left| \begin{array}{l}
%\hspace{5mm}\text{Définir le tour du joueur courant.} \\
%\hspace{5mm}\text{Afficher l'état des flottes.} \\
%\hspace{5mm}\text{Choisir le vaisseau attaquant puis le vaisseau ciblé.} \\
%\hspace{5mm}\text{Choisir le type d'attaque.} \\
%\hspace{5mm}\text{Résoudre l'attaque.} \\
%\hspace{2mm}\text{\bf Jusqu'à (\textit{destruction d'une des 2 flottes})}
%\end{array} \right.
%\end{equation*}
%\textbf{FinRépéter}
%
%\textbf{Afficher le vainqueur.}
%
%\end{algorithme}
%
%\item[$\looparrowright$] La mise en œuvre de requiert d'ajouter au conteneur \texttt{Flotte} une fonctionnalité \texttt{ChoixVaisseau} qui doit récupérer (en entrée/sortie) un pointeur vers le \texttt{Vaisseau}, \texttt{Croiseur} ou \texttt{Chasseur} en fonction du \textbf{numéro d'ordre} du vaisseau dans la flotte donné en entrée.\\
%\bfcolor{redCM}{Il faudra créer 3 surcharges de cette fonctionnalité pour qu'elle puisse récupérer, suivant le cas, un pointeur vers l'un des 3 types de \texttt{Vaisseau}.}
%%\itcolor{redCM}{\underline{Note :} la désignation d'un \textnormal{\tt Vaisseau}, \textnormal{\tt Croiseur} ou \textnormal{\tt Chasseur} au sein de la \texttt{Flotte} sera gérée par un \bfcolor{red}{numéro d'ordre}~:
%%\begin{itemize}
%%\item $\{1, 2, 3\}$ pour l'un des 3 \textnormal{\tt Vaisseau}.
%%\item $\{4, 5\}$ pour l'un des 2 \textnormal{\tt Chasseur}.
%%\item $\{6\}$ pour le \textnormal{\tt Croiseur}.
%%\end{itemize}
%%Ce \bfcolor{red}{numéro d'ordre} (type \textnormal{\tt int}) sera le paramètre d'entrée de la fonctionnalité \textnormal{\tt ChoixVaisseau}. Dans le programme principal, il faudra donc demander à l'utilisateur de choisir le numéro du vaisseau de la \textnormal{\tt Flotte} et récupérer le "bon" pointeur.}
%\item[$\looparrowright$] Dans un premier temps, on devra \bfcolor{greenC2}{tester le système de choix de vaisseau} en proposant une version  du programme de duel permettant définir 2 flottes et d'organiser un duel entre 2 vaisseaux de ces flottes.
%\item[$\looparrowright$] Enfin, on mettra en place tout \bfcolor{blueTD}{le programme du jeu}.
%\end{itemize}
%
%\cboxjalon{\bf Jalon \& Livrable n$^\circ$5 $\Rightarrow$ 10/06/2022}
%
%\cboxjalon{\bf OBJECTIF FINAL DU PROJET}
%
%\begin{mycolorbox}{black}{orange}
%Si vous arrivez jusqu'à ce jalon et que votre programme est totalement opérationnel (donc \og{}jouable\fg{}), \textbf{vous êtes les rois du monde}~!
%
%Pour ce jalon, vous devrez fournir dans une archive \texttt{ZIP} intitulée\\ \colorbox{yellow}{\texttt{Jalon\_5\_\textit{n$^\circ$binome}.zip}}~:
%\begin{itemize}
%\item[$\looparrowright$] Votre code fonctionnel et correctement documenté.
%\item[$\looparrowright$] La documentation technique du code, au format numérique PDF.
%\item[$\looparrowright$] Vos journaux de bord, personnel et de groupe, au format numérique PDF.
%\item[$\looparrowright$] \colorbox{yellow}{\bfcolor{black}{Une soutenance de votre projet sous forme d'une vidéo de 15 minutes.}} \itcolor{black}{Les modalités seront explicitées plus tard.}
%\end{itemize}
%\begin{mycolorbox}{black}{yellow}Si vous n'êtes pas parvenus jusqu'à ce point, vous devez rendre \textbf{votre projet en l'état} à la date prévue.\end{mycolorbox}
%\end{mycolorbox}
%
%\addtocounter{quest}{1}
%\subsection*{Phase~\thequest : quelques objectifs bonus (facultatifs)}
%\label{phase_objectifs_bonus}
%
%\begin{itemize}
%\item[$\looparrowright$] On propose dans cette phase quelques pistes d'exploration permettant de profiter de ce projet pour  \bfcolor{redCM}{faire monter vos compétences de programmeurs}.
%\item[$\looparrowright$] Ces \bfcolor{redCM}{objectifs} sont \bfcolor{redCM}{facultatifs} mais feront l'objet d'une \bfcolor{redCM}{bonification} dans l'évaluation de votre projet~: ils sont à explorer en totale liberté et autonomie.\\ \bfcolor{blueTD}{Vous êtes également libres d'explorer d'autres pistes suivant où vous porte votre curiosité \og{}informatique\fg{}.}\\
%\itcolor{redCM}{\bf Vous devrez vous-mêmes rechercher les ressources (tutoriels, documentations) pour vous aider à comprendre l'objectif et le mettre en œuvre dans votre programme.}
%\item[$\looparrowright$] Voici la liste (non exhaustive) des objectifs proposés~:
%\end{itemize}
%
%\begin{mycolorbox}{black}{yellow}
%\begin{itemize}
%\item[\ding{223}] \bfcolor{black}{Étudier la possibilité de créer des flottes personnalisées (nombre de vaisseaux, choix des types de vaisseaux) en utilisant notamment le principe d'allocation dynamique de mémoire.}
%\item[\ding{223}] \bfcolor{black}{Explorer la notion de \og{}pointeur générique\fg{} (\textit{void pointer} en anglais) \texttt{void*} pour faciliter la gestion de choix de vaisseaux dans un conteneur \texttt{Flotte} (en évitant trop de surcharges).}
%\item[\ding{223}] \bfcolor{black}{Explorer la gestion de fichiers (création/ouverture/lecture/écriture).} Par exemple~:
%\begin{itemize}
%\item écrire dans un fichier une partie de la sortie d'exécution (par exemple l'état de la partie à chaque tour)
%\item inventorier dans un ou plusieurs fichiers toutes les \og{}parties jouées\fg{} (à chaque exécution du programme) en précisant (suggestion) : le nom des joueurs, la date de la partie, sa durée, le nombre de tours de jeu, le vainqueur, etc.
%\item écrire dans des fichiers au préalable les caractéristiques de chaque vaisseau d'une flotte, puis initialiser les flottes en allant lire ces caractéristiques dans les fichiers.
%\end{itemize}
%\item[\ding{223}] \bfcolor{black}{Améliorer le moteur de jeu gérant la visée et le tir.} Par exemple~:
%\begin{itemize}
%\item proposer un affichage graphique plus élégant du système de visée mais toujours en mode \og{}terminal\fg{}~;
%\item \bfcolor{black}{aller plus loin en utilisant d'autres librairies/API (en langage C) plus graphiques pour gérer ce moteur.} Par exemple~:
%\begin{itemize}
%\item l'API \bfcolor{redCM}{ncurses}~: [\url{https://fr.wikipedia.org/wiki/Ncurses}]~; [\url{https://invisible-island.net/ncurses/}].
%\item la \bfcolor{redCM}{Simple DirectMedia Layer (SDL)}~: [\url{https://fr.wikipedia.org/wiki/Simple_DirectMedia_Layer}] ; [\url{https://zestedesavoir.com/tutoriels/1014/utiliser-la-sdl-en-langage-c/la-sdl/}] ; [\url{https://www.libsdl.org/}].
%\end{itemize}
%\end{itemize}
%\item[\ding{223}] Explorer l'utilisation d'un outil de versionnage de code tel que \bfcolor{redCM}{Git}, notamment via la plateforme (type \og{}cloud\fg{}) dédiée \bfcolor{redCM}{GitHub}~: [\url{https://git-scm.com/}]~;[\url{https://github.com/}].
%\item[\ding{223}] Explorer l'utilisation d'un outil de génération de documentation à partir des commentaires d'un code tel que \bfcolor{redCM}{Doxygen}~: [\url{https://www.doxygen.nl/index.html}].
%\item[\ding{223}] \bfcolor{black}{N'hésitez pas à explorer d'autres pistes qui vous intéressent ou à proposer d'autres améliorations du programme (autre système de jeu, enrichissement des types de vaisseaux, etc.).} 
%\end{itemize}
%\end{mycolorbox}
%
%\begin{mycolorbox}{black}{yellow}
%\large \bf Certains de ces objectifs annexes peuvent être traités en parallèle de l'élaboration du projet. Vous pouvez organiser votre travail d'équipe pour que certains d'entre vous se dédient à ces tâches. \bfcolor{redCM}{Quoi qu'il en soit, les phases 1 à 10 restant les objectifs prioritaires et obligatoires.}
%\end{mycolorbox}


\end{document}

